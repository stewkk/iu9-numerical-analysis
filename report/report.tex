% !TeX TXS-program:bibliography = txs:///biber
\documentclass[14pt, russian]{scrartcl}
\let\counterwithout\relax
\let\counterwithin\relax
%\usepackage{lmodern}
\usepackage{float}
\usepackage{xcolor}
\usepackage{extsizes}
\usepackage{subfig}
\usepackage[export]{adjustbox}
\usepackage{tocvsec2} % возможность менять учитываемую глубину разделов в оглавлении
\usepackage[subfigure]{tocloft}
\usepackage[newfloat,outputdir=build]{minted}
\captionsetup[listing]{position=top}

\AtBeginEnvironment{figure}{\vspace{0.5cm}}
\AtBeginEnvironment{table}{\vspace{0.5cm}}
\AtBeginEnvironment{listing}{\vspace{0.5cm}}
\AtBeginEnvironment{algorithm}{\vspace{0.5cm}}
\AtBeginEnvironment{minted}{\vspace{-0.5cm}}

\usepackage{fancyvrb}
\usepackage{ulem,bm,mathrsfs,ifsym} %зачеркивания, особо жирный стиль и RSFS начертание
\usepackage{sectsty} % переопределение стилей подразделов
%%%%%%%%%%%%%%%%%%%%%%%

%%% Поля и разметка страницы %%%
\usepackage{pdflscape}                              % Для включения альбомных страниц
\usepackage{geometry}                               % Для последующего задания полей
\geometry{a4paper,tmargin=2cm,bmargin=2cm,lmargin=3cm,rmargin=1cm} % тоже самое, но лучше

%%% Математические пакеты %%%
\usepackage{amsthm,amsfonts,amsmath,amssymb,amscd}  % Математические дополнения от AMS
\usepackage{mathtools}                              % Добавляет окружение multlined
\usepackage[perpage]{footmisc}
%\usepackage{times}

%%%% Установки для размера шрифта 14 pt %%%%
%% Формирование переменных и констант для сравнения (один раз для всех подключаемых файлов)%%
%% должно располагаться до вызова пакета fontspec или polyglossia, потому что они сбивают его работу
%\newlength{\curtextsize}
%\newlength{\bigtextsize}
%\setlength{\bigtextsize}{13pt}
\KOMAoptions{fontsize=14pt}

\makeatletter
\def\showfontsize{\f@size{} point}
\makeatother

%\makeatletter
%\show\f@size                                       % неплохо для отслеживания, но вызывает стопорение процесса, если документ компилируется без команды  -interaction=nonstopmode
%\setlength{\curtextsize}{\f@size pt}
%\makeatother

%шрифт times
\usepackage{tempora}
%\usepackage{pscyr}
%\setmainfont[Ligatures={TeX,Historic}]{Times New Roman}

   %%% Решение проблемы копирования текста в буфер кракозябрами
%    \input glyphtounicode.tex
%    \input glyphtounicode-cmr.tex %from pdfx package
%    \pdfgentounicode=1
    \usepackage{cmap}                               % Улучшенный поиск русских слов в полученном pdf-файле
    \usepackage[T1]{fontenc}                       % Поддержка русских букв
    \usepackage[utf8]{inputenc}                     % Кодировка utf8
    \usepackage[english, main=russian]{babel}            % Языки: русский, английский
%   \IfFileExists{pscyr.sty}{\usepackage{pscyr}}{}  % Красивые русские шрифты
%\renewcommand{\rmdefault}{ftm}
%%% Оформление абзацев %%%
\usepackage{indentfirst}                            % Красная строка
%\usepackage{eskdpz}

%%% Таблицы %%%
\usepackage{longtable}                              % Длинные таблицы
\usepackage{multirow,makecell,array}                % Улучшенное форматирование таблиц
\usepackage{booktabs}                               % Возможность оформления таблиц в классическом книжном стиле (при правильном использовании не противоречит ГОСТ)

%%% Общее форматирование
\usepackage{soulutf8}                               % Поддержка переносоустойчивых подчёркиваний и зачёркиваний
\usepackage{icomma}                                 % Запятая в десятичных дробях



%%% Изображения %%%
\usepackage{graphicx}                               % Подключаем пакет работы с графикой
\usepackage{wrapfig}

\usepackage{tikz}
\usetikzlibrary{shapes.misc}
\usetikzlibrary{trees}

%%% Списки %%%
\usepackage{enumitem}

%%% Подписи %%%
\usepackage{caption}                                % Для управления подписями (рисунков и таблиц) % Может управлять номерами рисунков и таблиц с caption %Иногда может управлять заголовками в списках рисунков и таблиц
%% Использование:
%\begin{table}[h!]\ContinuedFloat - чтобы не переключать счетчик
%\captionsetup{labelformat=continued}% должен стоять до самого caption
%\caption{}
% либо ручками \caption*{Продолжение таблицы~\ref{...}.} :)

%%% Интервалы %%%
\addto\captionsrussian{%
  \renewcommand{\listingname}{Листинг}%
}
%%% Счётчики %%%
\usepackage[figure,table,section]{totalcount}               % Счётчик рисунков и таблиц
\DeclareTotalCounter{lstlisting}
\usepackage{totcount}                               % Пакет создания счётчиков на основе последнего номера подсчитываемого элемента (может требовать дважды компилировать документ)
\usepackage{totpages}                               % Счётчик страниц, совместимый с hyperref (ссылается на номер последней страницы). Желательно ставить последним пакетом в преамбуле

%%% Продвинутое управление групповыми ссылками (пока только формулами) %%%
%% Кодировки и шрифты %%%

%   \newfontfamily{\cyrillicfont}{Times New Roman}
%   \newfontfamily{\cyrillicfonttt}{CMU Typewriter Text}
	%\setmainfont{Times New Roman}
	%\newfontfamily\cyrillicfont{Times New Roman}
	%\setsansfont{Times New Roman}                    %% задаёт шрифт без засечек
%	\setmonofont{Liberation Mono}               %% задаёт моноширинный шрифт
%    \IfFileExists{pscyr.sty}{\renewcommand{\rmdefault}{ftm}}{}
%%% Интервалы %%%
%linespread-реализация ближе к реализации полуторного интервала в ворде.
%setspace реализация заточена под шрифты 10, 11, 12pt, под остальные кегли хуже, но всё же ближе к типографской классике.
\linespread{1.3}                    % Полуторный интервал (ГОСТ Р 7.0.11-2011, 5.3.6)
%\renewcommand{\@biblabel}[1]{#1}

%%% Гиперссылки %%%
\usepackage{hyperref}

%%% Выравнивание и переносы %%%
\sloppy                             % Избавляемся от переполнений
\clubpenalty=10000                  % Запрещаем разрыв страницы после первой строки абзаца
\widowpenalty=10000                 % Запрещаем разрыв страницы после последней строки абзаца

\makeatletter % малые заглавные, small caps shape
\let\@@scshape=\scshape
\renewcommand{\scshape}{%
  \ifnum\strcmp{\f@series}{bx}=\z@
    \usefont{T1}{cmr}{bx}{sc}%
  \else
    \ifnum\strcmp{\f@shape}{it}=\z@
      \fontshape{scsl}\selectfont
    \else
      \@@scshape
    \fi
  \fi}
\makeatother

%%% Подписи %%%
%\captionsetup{%
%singlelinecheck=off,                % Многострочные подписи, например у таблиц
%skip=2pt,                           % Вертикальная отбивка между подписью и содержимым рисунка или таблицы определяется ключом
%justification=centering,            % Центрирование подписей, заданных командой \caption
%}
%%%        Подключение пакетов                 %%%
\usepackage{ifthen}                 % добавляет ifthenelse
%%% Инициализирование переменных, не трогать!  %%%
\newcounter{intvl}
\newcounter{otstup}
\newcounter{contnumeq}
\newcounter{contnumfig}
\newcounter{contnumtab}
\newcounter{pgnum}
\newcounter{bibliosel}
\newcounter{chapstyle}
\newcounter{headingdelim}
\newcounter{headingalign}
\newcounter{headingsize}
\newcounter{tabcap}
\newcounter{tablaba}
\newcounter{tabtita}
%%%%%%%%%%%%%%%%%%%%%%%%%%%%%%%%%%%%%%%%%%%%%%%%%%

%%% Область упрощённого управления оформлением %%%

%% Интервал между заголовками и между заголовком и текстом
% Заголовки отделяют от текста сверху и снизу тремя интервалами (ГОСТ Р 7.0.11-2011, 5.3.5)
\setcounter{intvl}{3}               % Коэффициент кратности к размеру шрифта

%% Отступы у заголовков в тексте
\setcounter{otstup}{0}              % 0 --- без отступа; 1 --- абзацный отступ

%% Нумерация формул, таблиц и рисунков
\setcounter{contnumeq}{1}           % Нумерация формул: 0 --- пораздельно (во введении подряд, без номера раздела); 1 --- сквозная нумерация по всей диссертации
\setcounter{contnumfig}{1}          % Нумерация рисунков: 0 --- пораздельно (во введении подряд, без номера раздела); 1 --- сквозная нумерация по всей диссертации
\setcounter{contnumtab}{1}          % Нумерация таблиц: 0 --- пораздельно (во введении подряд, без номера раздела); 1 --- сквозная нумерация по всей диссертации

%% Оглавление
\setcounter{pgnum}{0}               % 0 --- номера страниц никак не обозначены; 1 --- Стр. над номерами страниц (дважды компилировать после изменения)

%% Библиография
\setcounter{bibliosel}{1}           % 0 --- встроенная реализация с загрузкой файла через движок bibtex8; 1 --- реализация пакетом biblatex через движок biber

%% Текст и форматирование заголовков
\setcounter{chapstyle}{1}           % 0 --- разделы только под номером; 1 --- разделы с названием "Глава" перед номером
\setcounter{headingdelim}{1}        % 0 --- номер отделен пропуском в 1em или \quad; 1 --- номера разделов и приложений отделены точкой с пробелом, подразделы пропуском без точки; 2 --- номера разделов, подразделов и приложений отделены точкой с пробелом.

%% Выравнивание заголовков в тексте
\setcounter{headingalign}{0}        % 0 --- по центру; 1 --- по левому краю

%% Размеры заголовков в тексте
\setcounter{headingsize}{0}         % 0 --- по ГОСТ, все всегда 14 пт; 1 --- пропорционально изменяющийся размер в зависимости от базового шрифта

%% Подпись таблиц
\setcounter{tabcap}{0}              % 0 --- по ГОСТ, номер таблицы и название разделены тире, выровнены по левому краю, при необходимости на нескольких строках; 1 --- подпись таблицы не по ГОСТ, на двух и более строках, дальнейшие настройки:
%Выравнивание первой строки, с подписью и номером
\setcounter{tablaba}{2}             % 0 --- по левому краю; 1 --- по центру; 2 --- по правому краю
%Выравнивание строк с самим названием таблицы
\setcounter{tabtita}{1}             % 0 --- по левому краю; 1 --- по центру; 2 --- по правому краю

%%% Рисунки %%%
\DeclareCaptionLabelSeparator*{emdash}{~--- }             % (ГОСТ 2.105, 4.3.1)
\captionsetup[figure]{labelsep=emdash,font=onehalfspacing,position=bottom}

%%% Таблицы %%%
\ifthenelse{\equal{\thetabcap}{0}}{%
    \newcommand{\tabcapalign}{\raggedright}  % по левому краю страницы или аналога parbox
}

\ifthenelse{\equal{\thetablaba}{0} \AND \equal{\thetabcap}{1}}{%
    \newcommand{\tabcapalign}{\raggedright}  % по левому краю страницы или аналога parbox
}

\ifthenelse{\equal{\thetablaba}{1} \AND \equal{\thetabcap}{1}}{%
    \newcommand{\tabcapalign}{\centering}    % по центру страницы или аналога parbox
}

\ifthenelse{\equal{\thetablaba}{2} \AND \equal{\thetabcap}{1}}{%
    \newcommand{\tabcapalign}{\raggedleft}   % по правому краю страницы или аналога parbox
}

\ifthenelse{\equal{\thetabtita}{0} \AND \equal{\thetabcap}{1}}{%
    \newcommand{\tabtitalign}{\raggedright}  % по левому краю страницы или аналога parbox
}

\ifthenelse{\equal{\thetabtita}{1} \AND \equal{\thetabcap}{1}}{%
    \newcommand{\tabtitalign}{\centering}    % по центру страницы или аналога parbox
}

\ifthenelse{\equal{\thetabtita}{2} \AND \equal{\thetabcap}{1}}{%
    \newcommand{\tabtitalign}{\raggedleft}   % по правому краю страницы или аналога parbox
}

\DeclareCaptionFormat{tablenocaption}{\tabcapalign #1\strut}        % Наименование таблицы отсутствует
\ifthenelse{\equal{\thetabcap}{0}}{%
    \DeclareCaptionFormat{tablecaption}{\tabcapalign #1#2#3}
    \captionsetup[table]{labelsep=emdash}                       % тире как разделитель идентификатора с номером от наименования
}{%
    \DeclareCaptionFormat{tablecaption}{\tabcapalign #1#2\par%  % Идентификатор таблицы на отдельной строке
        \tabtitalign{#3}}                                       % Наименование таблицы строкой ниже
    \captionsetup[table]{labelsep=space}                        % пробельный разделитель идентификатора с номером от наименования
}
\captionsetup[table]{format=tablecaption,singlelinecheck=off,font=onehalfspacing,position=top,skip=-5pt}  % многострочные наименования и прочее
\DeclareCaptionLabelFormat{continued}{Продолжение таблицы~#2}
\setlength{\belowcaptionskip}{.2cm}
\setlength{\intextsep}{0ex}

%%% Подписи подрисунков %%%
\renewcommand{\thesubfigure}{\asbuk{subfigure}}           % Буквенные номера подрисунков
\captionsetup[subfigure]{font={normalsize},               % Шрифт подписи названий подрисунков (не отличается от основного)
    labelformat=brace,                                    % Формат обозначения подрисунка
    justification=centering,                              % Выключка подписей (форматирование), один из вариантов
}
%\DeclareCaptionFont{font12pt}{\fontsize{12pt}{13pt}\selectfont} % объявляем шрифт 12pt для использования в подписях, тут же надо интерлиньяж объявлять, если не наследуется
%\captionsetup[subfigure]{font={font12pt}}                 % Шрифт подписи названий подрисунков (всегда 12pt)

%%% Настройки гиперссылок %%%

\definecolor{linkcolor}{rgb}{0.0,0,0}
\definecolor{citecolor}{rgb}{0,0.0,0}
\definecolor{urlcolor}{rgb}{0,0,0}

\hypersetup{
    linktocpage=true,           % ссылки с номера страницы в оглавлении, списке таблиц и списке рисунков
%    linktoc=all,                % both the section and page part are links
%    pdfpagelabels=false,        % set PDF page labels (true|false)
    plainpages=true,           % Forces page anchors to be named by the Arabic form  of the page number, rather than the formatted form
    colorlinks,                 % ссылки отображаются раскрашенным текстом, а не раскрашенным прямоугольником, вокруг текста
    linkcolor={linkcolor},      % цвет ссылок типа ref, eqref и подобных
    citecolor={citecolor},      % цвет ссылок-цитат
    urlcolor={urlcolor},        % цвет гиперссылок
    pdflang={ru},
}
\urlstyle{same}
%%% Шаблон %%%
%\DeclareRobustCommand{\todo}{\textcolor{red}}       % решаем проблему превращения названия цвета в результате \MakeUppercase, http://tex.stackexchange.com/a/187930/79756 , \DeclareRobustCommand protects \todo from expanding inside \MakeUppercase
\setlength{\parindent}{2.5em}                       % Абзацный отступ. Должен быть одинаковым по всему тексту и равен пяти знакам (ГОСТ Р 7.0.11-2011, 5.3.7).

%%% Списки %%%
% Используем дефис для ненумерованных списков (ГОСТ 2.105-95, 4.1.7)
%\renewcommand{\labelitemi}{\normalfont\bfseries~{---}}
\renewcommand{\labelitemi}{\bfseries~{---}}
\setlist{nosep,%                                    % Единый стиль для всех списков (пакет enumitem), без дополнительных интервалов.
    labelindent=\parindent,leftmargin=*%            % Каждый пункт, подпункт и перечисление записывают с абзацного отступа (ГОСТ 2.105-95, 4.1.8)
}
%%%%%%%%%%%%%%%%%%%%%%
%\usepackage{xltxtra} % load xunicode

\usepackage{ragged2e}
\usepackage[explicit]{titlesec}
\usepackage{placeins}
\usepackage{xparse}
\usepackage{csquotes}

\usepackage{listingsutf8}
\usepackage{url} %пакеты расширений
\usepackage{algorithm, algorithmicx}
\usepackage[noend]{algpseudocode}
\usepackage{blkarray}
\usepackage{chngcntr}
\usepackage{tabularx}
\usepackage[backend=biber,
    bibstyle=gost-numeric,
    citestyle=nature]{biblatex}
\newcommand*\template[1]{\text{<}#1\text{>}}
\addbibresource{biblio.bib}

\titleformat{name=\section,numberless}[block]{\normalfont\Large\centering}{}{0em}{#1}
\titleformat{\section}[block]{\normalfont\Large\bfseries\raggedright}{}{0em}{\thesection\hspace{0.25em}#1}
\titleformat{\subsection}[block]{\normalfont\Large\bfseries\raggedright}{}{0em}{\thesubsection\hspace{0.25em}#1}
\titleformat{\subsubsection}[block]{\normalfont\large\bfseries\raggedright}{}{0em}{\thesubsubsection\hspace{0.25em}#1}

\let\Algorithm\algorithm
\renewcommand\algorithm[1][]{\Algorithm[#1]\setstretch{1.5}}
%\renewcommand{\listingscaption}{Листинг}

\usepackage{pifont}
\usepackage{calc}
\usepackage{suffix}
\usepackage{csquotes}
\DeclareQuoteStyle{russian}
    {\guillemotleft}{\guillemotright}[0.025em]
    {\quotedblbase}{\textquotedblleft}
\ExecuteQuoteOptions{style=russian}
\newcommand{\enq}[1]{\enquote{#1}}
\newcommand{\eng}[1]{\begin{english}#1\end{english}}
% Подчиненные счетчики в окружениях http://old.kpfu.ru/journals/izv_vuz/arch/sample1251.tex
\newcounter{cTheorem}
\newcounter{cDefinition}
\newcounter{cConsequent}
\newcounter{cExample}
\newcounter{cLemma}
\newcounter{cConjecture}
\newtheorem{Theorem}{Теорема}[cTheorem]
\newtheorem{Definition}{Определение}[cDefinition]
\newtheorem{Consequent}{Следствие}[cConsequent]
\newtheorem{Example}{Пример}[cExample]
\newtheorem{Lemma}{Лемма}[cLemma]
\newtheorem{Conjecture}{Гипотеза}[cConjecture]

\renewcommand{\theTheorem}{\arabic{Theorem}}
\renewcommand{\theDefinition}{\arabic{Definition}}
\renewcommand{\theConsequent}{\arabic{Consequent}}
\renewcommand{\theExample}{\arabic{Example}}
\renewcommand{\theLemma}{\arabic{Lemma}}
\renewcommand{\theConjecture}{\arabic{Conjecture}}
%\makeatletter
\NewDocumentCommand{\Newline}{}{\text{\\}}
\newcommand{\sequence}[2]{\ensuremath \left(#1,\ \dots,\ #2\right)}

\definecolor{mygreen}{rgb}{0,0.6,0}
\definecolor{mygray}{rgb}{0.5,0.5,0.5}
\definecolor{mymauve}{rgb}{0.58,0,0.82}
\renewcommand{\listalgorithmname}{Список алгоритмов}
\floatname{algorithm}{Листинг}
\renewcommand{\lstlistingname}{Листинг}
\renewcommand{\thealgorithm}{\arabic{algorithm}}

\newcommand{\refAlgo}[1]{(листинг~\ref{#1})}
\newcommand{\refImage}[1]{(рисунок~\ref{#1})}

\renewcommand{\theenumi}{\arabic{enumi}.}% Меняем везде перечисления на цифра.цифра
\renewcommand{\labelenumi}{\arabic{enumi}.}% Меняем везде перечисления на цифра.цифра
\renewcommand{\theenumii}{\arabic{enumii}}% Меняем везде перечисления на цифра.цифра
\renewcommand{\labelenumii}{(\arabic{enumii})}% Меняем везде перечисления на цифра.цифра
\renewcommand{\theenumiii}{\roman{enumiii}}% Меняем везде перечисления на цифра.цифра
\renewcommand{\labelenumiii}{(\roman{enumiii})}% Меняем везде перечисления на цифра.цифра
%\newfontfamily\AnkaCoder[Path=src/fonts/]{AnkaCoder-r.ttf}
\renewcommand{\labelitemi}{---}
\renewcommand{\labelitemii}{---}

%\usepackage{courier}

\lstdefinelanguage{Refal}{
  alsodigit = {.,<,>},
  morekeywords = [1]{$ENTRY},
  morekeywords = [2]{Go, Put, Get, Open, Close, Arg, Add, Sub, Mul, Div, Symb, Explode, Implode},
  %keyword4
  morekeywords = [3]{<,>},
  %keyword5
  morekeywords = [4]{e.,t.,s.},
  sensitive = true,
  morecomment = [l]{*},
  morecomment = [s]{/*}{*/},
  commentstyle = \color{mygreen},
  morestring = [b]",
  morestring = [b]',
  stringstyle = \color{purple}
}

\makeatletter
\def\p@subsection{}
\def\p@subsubsection{\thesection\,\thesubsection\,}
\makeatother
\newcommand{\prog}[1]{{\ttfamily\small#1}}
\lstset{ %
  backgroundcolor=\color{white},   % choose the background color; you must add \usepackage{color} or \usepackage{xcolor}
  basicstyle=\ttfamily\footnotesize,
  %basicstyle=\footnotesize\AnkaCoder,        % the size of the fonts that are used for the code
  breakatwhitespace=false,         % sets if automatic breaks shoulbd only happen at whitespace
  breaklines=true,                 % sets automatic line breaking
  captionpos=top,                    % sets the caption-position to bottom
  commentstyle=\color{mygreen},    % comment style
  deletekeywords={...},            % if you want to delete keywords from the given language
  escapeinside={\%*}{*)},          % if you want to add LaTeX within your code
  extendedchars=true,              % lets you use non-ASCII characters; for 8-bits encodings only, does not work with UTF-8
  inputencoding=utf8,
  frame=single,                    % adds a frame around the code
  keepspaces=true,                 % keeps spaces in text, useful for keeping indentation of code (possibly needs columns=flexible)
  keywordstyle=\bf,       % keyword style
  language=Refal,                    % the language of the code
  morekeywords={<,>,$ENTRY,Go,Arg, Open, Close, e., s., t., Get, Put},
  							       % if you want to add more keywords to the set
  numbers=left,                    % where to put the line-numbers; possible values are (none, left, right)
  numbersep=5pt,                   % how far the line-numbers are from the code
  xleftmargin=25pt,
  xrightmargin=25pt,
  numberstyle=\small\color{black}, % the style that is used for the line-numbers
  rulecolor=\color{black},         % if not set, the frame-color may be changed on line-breaks within not-black text (e.g. comments (green here))
  showspaces=false,                % show spaces everywhere adding particular underscores; it overrides 'showstringspaces'
  showstringspaces=false,          % underline spaces within strings only
  showtabs=false,                  % show tabs within strings adding particular underscores
  stepnumber=1,                    % the step between two line-numbers. If it's 1, each line will be numbered
  stringstyle=\color{mymauve},     % string literal style
  tabsize=8,                       % sets default tabsize to 8 spaces
  title=\lstname                   % show the filename of files included with \lstinputlisting; also try caption instead of title
}
\newcommand{\anonsection}[1]{\cleardoublepage
\phantomsection
\addcontentsline{toc}{section}{\protect\numberline{}#1}
\section*{#1}\vspace*{2.5ex} % По госту положены 3 пустые строки после заголовка ненумеруемого раздела
}
\newcommand{\sectionbreak}{\clearpage}
\renewcommand{\sectionfont}{\normalsize} % Сбиваем стиль оглавления в стандартный
\renewcommand{\cftsecleader}{\cftdotfill{\cftdotsep}} % Точки в оглавлении напротив разделов

\renewcommand{\cftsecfont}{\normalfont\large} % Переключение на times в содержании
\renewcommand{\cftsubsecfont}{\normalfont\large} % Переключение на times в содержании

\usepackage{caption}
%\captionsetup[table]{justification=raggedleft}
%\captionsetup[figure]{justification=centering,labelsep=endash}
\usepackage{amsmath}    % \bar    (матрицы и проч. ...)
\usepackage{amsfonts}   % \mathbb (символ для множества действительных чисел и проч. ...)
\usepackage{mathtools}  % \abs, \norm
    \DeclarePairedDelimiter\abs{\lvert}{\rvert} % операция модуля
    \DeclarePairedDelimiter\norm{\lVert}{\rVert} % операция нормы
\DeclareTextCommandDefault{\textvisiblespace}{%
  \mbox{\kern.06em\vrule \@height.3ex}%
  \vbox{\hrule \@width.3em}%
  \hbox{\vrule \@height.3ex}}
\newsavebox{\spacebox}
\begin{lrbox}{\spacebox}
\verb*! !
\end{lrbox}
\newcommand{\aspace}{\usebox{\spacebox}}
\DeclareTotalCounter{listing}

\makeatletter
\renewcommand*{\p@subsubsection}{}
\makeatother

\makeatletter
\AddToHook{begindocument/before}{\@ifpackageloaded{minted}{\removefromtoclist[float]{lol}}{}}
\makeatother

\begin{document}
\sloppy

\def\figurename{Рисунок}

\begin{titlepage}
	\thispagestyle{empty}
	\newpage

	\vspace*{-30pt}
	\hspace{-45pt}
	\begin{minipage}{0.17\textwidth}
		\hspace*{-20pt}\centering
		\includegraphics[width=1.3\textwidth]{emblem.png}
	\end{minipage}
	\begin{minipage}{0.82\textwidth}\small \textbf{
			\vspace*{-0.7ex}
			\hspace*{-10pt}\centerline{Министерство науки и высшего образования Российской Федерации}
			\vspace*{-0.7ex}
			\centerline{Федеральное государственное автономное образовательное учреждение }
			\vspace*{-0.7ex}
			\centerline{высшего образования}
			\vspace*{-0.7ex}
			\centerline{<<Московский государственный технический университет}
			\vspace*{-0.7ex}
			\centerline{имени Н.Э. Баумана}
			\vspace*{-0.7ex}
			\centerline{(национальный исследовательский университет)>>}
			\vspace*{-0.7ex}
			\centerline{(МГТУ им. Н.Э. Баумана)}}
	\end{minipage}

	\vspace{-2pt}
	\hspace{-34.5pt}\rule{\textwidth}{2.5pt}

	\vspace*{-20.3pt}
	\hspace{-34.5pt}\rule{\textwidth}{0.4pt}

	\vspace{0.5ex}
	\noindent \small ФАКУЛЬТЕТ\hspace{80pt} <<Информатика и системы управления>>

	\vspace*{-16pt}
	\hspace{35pt}\rule{0.855\textwidth}{0.4pt}

	\vspace{0.5ex}
	\noindent \small КАФЕДРА\hspace{50pt} <<Теоретическая информатика и компьютерные технологии>>

	\vspace*{-16pt}
	\hspace{25pt}\rule{0.875\textwidth}{0.4pt}


	\vspace{3em}

	\begin{center}
		\textbf{ОТЧЕТ} \\\textit{по лабораторной работе № 1\\по курсу <<Численные методы>>\\на тему: <<Сплайн-интерполяция>>\\Вариант № 26} \\
	\end{center}

	\vspace{\fill}


	\newlength{\ML}
	\settowidth{\ML}{«\underline{\hspace{0.7cm}}» \underline{\hspace{2cm}}}

	\noindent Студент \underline{\text{ИУ9-61Б}} \hfill \underline{ \hspace{4cm}}\quad
	\raisebox{0.45ex}{\underline{\parbox{4cm}{\centering Старовойтов А.И.}}}

	\vspace{-2.1ex}
	\noindent\hspace{9ex}\scriptsize{(Группа)}\normalsize\hspace{170pt}\hspace{2ex}\scriptsize{(Подпись, дата)}\normalsize\hspace{30pt}\hspace{6ex}\scriptsize{(И.О. Фамилия)}\normalsize

	\bigskip

	\noindent Преподаватель  \hfill \underline{\hspace{4cm}}\quad
	\raisebox{0.35ex}{\underline{\parbox{4cm}{\centering Домрачева А.Б.}}}

	\vspace{-2ex}
	\noindent\hspace{13.5ex}\normalsize\hspace{170pt}\hspace{2ex}\scriptsize{(Подпись, дата)}\normalsize\hspace{30pt}\hspace{6ex}\scriptsize{(И.О. Фамилия)}\normalsize

	\bigskip

	%\vspace{\fill}



	\begin{center}
		\textsl{2025 г.}
	\end{center}
\end{titlepage}

%\renewcommand{\ttdefault}{pcr}

\setlength{\tabcolsep}{3pt}
\newpage
\setcounter{page}{2}
%----------------------------------------------------------------------------
%                  ОТСЮДА --- СОБСТВЕННО ТЕКСТ
%----------------------------------------------------------------------------

\newpage
\section{Постановка задачи}

Протабулировать функцию $f(x)$ на отрезке $[a, b]$ с шагом $h = \frac{b-a}{32}$.
Распечатать таблицу $(x_{i}, y_{i}),$ где $i = 0, \ldots{}, n$. Для полученных
узлов построить кубический сплайн и распечатать для него массивы $a$, $b$, $c$,
$d$. Вычислить значения $f(x)$ в точках
$x_{i}^{*} = a + (i - \frac{1}{2}h), i = 1, \ldots{}, n$. Вычислить значения
исходной функции в указанных точках и сравнить полученные результаты.

Индивидуальный вариант: $f(x) = \sin{x} \cdot \cos{\frac{x}{2}}$ на $[0, \pi]$

\section{Теоретический раздел}

Интерполяционной называется функция $y = g(x)$, проходящая через заданные точки,
называемые узлами интерполяции ($g(x_{i}) = f(x_{i}), i = 0, \ldots{}, n$). При
этом в промежуточных точках равенство выполняется с некоторой погрешностью
$g(x_{i}^{*}) \approx f(x_{i}^{*})$. Задача интерполяции заключается в поиске
такой функции $y = g(x)$.

Приближение функции кубическим сплайном --- пример задачи интерполяции.

Сплайном $k$-го порядка с дефектом $def$ называется функция, проходящая через
все узлы $(x_{i}, y_{i}), i=0, \ldots{}, n$, являющаяся многочленом $k$-й
степени на каждом отрезке разбиения $[x_{i}, x_{i+1}]$ в отдельности и имеющая
$(k - def)$ непрерывных производных на отрезке $[x_{0}, x_{n}]$.

Для каждого отрезка разбиения отыскивается кубический сплайн в виде:

\begin{equation}
  S_{i}(x_{i}) = y_{i}, i=0, \ldots{}, n-1, S_{n-1}(x_{n}) = y_{n}
\end{equation}

\begin{equation}
  S_{i-1}(x_{i}) = S_{i}(x_{i}); S_{i-1}^{'}(x_{i}) = S_{i}^{'}(x_{i}); S_{i-1}^{''}(x_{i}) = S_{i}^{''}(x_{i}); i=0, \ldots{}, n-1
\end{equation}

\begin{equation}
  S_{0}^{''}(x_{0}) = 0; S_{n-1}^{''}(x_{n}) = 0
\end{equation}

Эти условия приводят к трехдиагональной СЛАУ относительно $c_{i}$:

\begin{equation}
  c_{i-1} + 4c_{i} + c_{i+1} = \frac{y_{i+1} - 2y_{i} + y_{i-1}}{h^{2}}, i=1, \ldots{}, n-1, c_{0} = c_{n} = 0
\end{equation}

где $h = x_{i+1} - x_{i}$ --- шаг интерполирования. Система затем решается
методом прогонки и оставшиеся коэффициенты вычисляются через $c_{i}$:

\begin{equation}
  a_{i} = y_{i}, i=0. \ldots{}, n-1;
\end{equation}
\begin{equation}
  b_{i} = \frac{y_{i+1} - y_{i}}{h} \frac{h}{3} (c_{i+1} + 2c_{i}), i=0, \ldots, n-2; b_{n-1} = \frac{y_{n} - y_{n-1}}{h} \frac{2}{3} h c_{n-1};
\end{equation}
\begin{equation}
  d_{i} = \frac{c_{i+1} - c_{i}}{2h}; i = 0, \ldots, n-2; d_{n-1} = - \frac{c_{n}}{3h}
\end{equation}

\section{Практический раздел}

\captionof{listing}{Исходный код программы на Python}
\vspace{0.5cm}
  \begin{minted}[style=bw, breaklines, frame=single, fontsize = \footnotesize, linenos=false, xleftmargin = 1.5em]{python}
#!/usr/bin/env python3

# Вариант 26

import math


def f(x):
    return math.sin(x) * math.cos(x/2)


def forward_pass(d: list[float], a: list[float], b: list[float], c: list[float]) -> tuple[list[float], list[float]]:
    alpha, beta = list(), list()
    alpha.append(-c[0] / b[0])
    beta.append(d[0] / b[0])
    n = len(d)
    for i in range(1, n-1):
        tmp = b[i] + a[i-1]*alpha[i-1]
        alpha.append(-c[i] / tmp)
        beta.append((d[i] - a[i-1]*beta[i-1]) / tmp)
    beta.append(
        (d[n-1] - a[n-2]*beta[n-2]) / (b[n-1] + a[n-2]*alpha[n-2]))
    return (alpha, beta)


def backward_pass(alpha: list[float], beta: list[float]) -> list[float]:
    n = len(beta)
    x = list(range(n))
    x[n-1] = beta[n-1]
    for i in range(n-2, -1, -1):
        x[i] = alpha[i]*x[i+1] + beta[i]
    return x


def calc_table(n, a, h):
    x = list()
    y = list()
    for i in range(0, n+1):
        x.append(a+h*i)
        y.append(f(x[-1]))
    return x, y


def calc_free_coefficients(y, n, h):
    res = list()
    res.append(0)
    for i in range(1, n):
        res.append(3*(y[i+1] - 2 * y[i] + y[i-1]) / h / h)
    return res


def main():
    l = 0
    r = math.pi
    n = 32
    h = (r-l) / n

    x, y = calc_table(n, l, h)
    for i in range(len(x)):
        print('x: ', '{0:0.15f}'.format(x[i]), 'y: ', '{0:0.15f}'.format(y[i]))
    print()

    free_coeffs = calc_free_coefficients(y, n, h)

    b = [4.0 for _ in range(n)]
    c = [1.0 for _ in range(n-1)]
    a = [1.0 for _ in range(n-1)]

    alpha, beta = forward_pass(free_coeffs, a, b, c)

    c_res = backward_pass(alpha, beta)
    c_res.insert(0, 0)
    c_res.append(0)

    a_res = [y[i] for i in range(len(y))]

    b_res = [(y[i+1] - y[i]) / h - (h/3) * (c_res[i+1] + 2 * c_res[i]) for i in range(0, n)]

    d_res = [(c_res[i+1] - c_res[i]) / 3 / h for i in range(0, n-1)]
    d_res.append(-c_res[n] / 3 / h)

    print('a: ', a_res)
    print('b: ', b_res)
    print('c: ', c_res)
    print('d: ', d_res)
    print()

    x_star = [l+(i-1/2)*h for i in range(1, n+1)]
    y_star = [f(x) for x in x_star]

    y_approx = list()
    diff = list()

    for i in range(n):
        y_approx.append( a_res[i] + b_res[i] * (x_star[i] - x[i]) + c_res[i] * ((x_star[i] - x[i]) ** 2) + d_res[i] * (
                    (x_star[i] - x[i]) ** 3))
        diff.append(math.fabs(y_approx[i] - y_star[i]))

    for i in range(n):
        print('x: {0:0.7f} y: {1:0.7f} y*: {2:0.7f} d*: {3:0.7f}'.format(x_star[i], y_star[i], y_approx[i], diff[i]))

main()
  \end{minted}


\section{Тестирование}

\captionof{listing}{Результаты работы программы}
\vspace{0.5cm}
  \begin{minted}[style=bw, breaklines, frame=single, fontsize = \footnotesize, linenos=false, xleftmargin = 1.5em]{md}
x:  0.000000000000000 y:  0.000000000000000
x:  0.098174770424681 y:  0.097899074391390
x:  0.196349540849362 y:  0.194150908792011
x:  0.294524311274043 y:  0.287142783942822
x:  0.392699081698724 y:  0.375330277517865
x:  0.490873852123405 y:  0.457269567375141
x:  0.589048622548086 y:  0.531647565308600
x:  0.687223392972767 y:  0.597309231696246
x:  0.785398163397448 y:  0.653281482438188
x:  0.883572933822129 y:  0.698793173312413
x:  0.981747704246810 y:  0.733290731749097
x:  1.079922474671491 y:  0.756449100199197
x:  1.178097245096172 y:  0.768177756711416
x:  1.276272015520854 y:  0.768621684837727
x:  1.374446785945534 y:  0.758157274256000
x:  1.472621556370215 y:  0.737383243163832
x:  1.570796326794897 y:  0.707106781186548
x:  1.668971097219578 y:  0.668325214923696
x:  1.767145867644259 y:  0.622203595094367
x:  1.865320638068940 y:  0.570048692436433
x:  1.963495408493621 y:  0.513279967159337
x:  2.061670178918302 y:  0.453398142163845
x:  2.159844949342983 y:  0.391952062009397
x:  2.258019719767664 y:  0.330504556610090
x:  2.356194490192345 y:  0.270598050073099
x:  2.454369260617026 y:  0.213720660494900
x:  2.552544031041707 y:  0.161273525784282
x:  2.650718801466388 y:  0.114540063919793
x:  2.748893571891069 y:  0.074657834050343
x:  2.847068342315750 y:  0.042593608420669
x:  2.945243112740431 y:  0.019122195469994
x:  3.043417883165112 y:  0.004809473120196
x:  3.141592653589793 y:  0.000000000000000

a:  [0.0, 0.09789907439138988, 0.19415090879201147, 0.2871427839428219, 0.37533027751786524, 0.45726956737514113, 0.5316475653085996, 0.5973092316962461, 0.6532814824381882, 0.6987931733124131, 0.7332907317490973, 0.756449100199197, 0.7681777567114163, 0.7686216848377272, 0.7581572742560003, 0.7373832431638316, 0.7071067811865476, 0.6683252149236961, 0.6222035950943673, 0.5700486924364326, 0.5132799671593368, 0.453398142163845, 0.3919520620093975, 0.3305045566100899, 0.27059805007309856, 0.21372066049489952, 0.16127352578428164, 0.11454006391979255, 0.07465783405034265, 0.042593608420668824, 0.019122195469994076, 0.004809473120195732, 7.498798913309288e-33]
b:  [0.9964398524331297, 0.9819169961605076, 0.9587225951950499, 0.9174690097013595, 0.8616656880044221, 0.7917714873264311, 0.7094376686173278, 0.6163327336758673, 0.5144117981229063, 0.40579263175885577, 0.2927319420605111, 0.17757087937723226, 0.06268618743980713, -0.04956138773049281, -0.15687778395111363, -0.25708498561378174, -0.34816781355435844, -0.4283171788251479, -0.49596885151322107, -0.5498369112815715, -0.5889411501130656, -0.612627855777461, -0.620583450925596, -0.6128409781149686, -0.5897782300944288, -0.5521115283703123, -0.5008705019662302, -0.437409588174874, -0.3632500932721746, -0.28052471621741, -0.19014689714942842, -0.09844820492672429]
c:  [0, 0.02297731367175994, -0.09190925468703975, -0.16805803375026268, -0.2505492107086729, -0.32515057240420725, -0.393656117177694, -0.4537426730705362, -0.5044070083676442, -0.5445439800007692, -0.5733570878350343, -0.5902719501006459, -0.5949763755916575, -0.58741955408477, -0.5678150261711173, -0.5366354064823399, -0.49460217958091923, -0.44266993269485233, -0.3820055900767887, -0.3139630629644081, -0.24005378413150033, -0.16191412093722846, -0.08126902918813504, 0.00010224775354838664, 0.0804164183456057, 0.15788134469611542, 0.230897137364891, 0.29748537093146615, 0.35759162494527313, 0.4046535300548087, 0.4572159046375046, 0.4410703502590185, 0.6292246060158955, 0]
d:  [0.07801499839644975, -0.3900749919822487, -0.2585483647642572, -0.2800827429205173, -0.2532944101382542, -0.23259724970460954, -0.20401221085255009, -0.1720208938879985, -0.13627727863788863, -0.09782930213681276, -0.05743112408086771, -0.01597296152110953, 0.02565771726686495, 0.06656336052140499, 0.10586433274014545, 0.14271564448990562, 0.1763258410194397, 0.20597397327115008, 0.23102516331519307, 0.2509445774211086, 0.2653080245099565, 0.2738147197433097, 0.2762803402873276, 0.2726911413344353, 0.26301708682524244, 0.24791091897618364, 0.22608739250939952, 0.2040790919900684, 0.1597895430294253, 0.17846531702365556, -0.05481908881696134, -2.136409735732909]

x: 0.0490874 y: 0.0490529 y*: 0.0489219 d*: 0.0001310
x: 0.1472622 y: 0.1463329 y*: 0.1461080 d*: 0.0002249
x: 0.2454369 y: 0.2411529 y*: 0.2409601 d*: 0.0001928
x: 0.3436117 y: 0.3319300 y*: 0.3317409 d*: 0.0001892
x: 0.4417865 y: 0.4171664 y*: 0.4169935 d*: 0.0001729
x: 0.5399612 y: 0.4954799 y*: 0.4953246 d*: 0.0001553
x: 0.6381360 y: 0.5656333 y*: 0.5654993 d*: 0.0001339
x: 0.7363108 y: 0.6265597 y*: 0.6264497 d*: 0.0001099
x: 0.8344855 y: 0.6773847 y*: 0.6773011 d*: 0.0000837
x: 0.9326603 y: 0.7174445 y*: 0.7173888 d*: 0.0000557
x: 1.0308351 y: 0.7462985 y*: 0.7462718 d*: 0.0000267
x: 1.1290099 y: 0.7637386 y*: 0.7637414 d*: 0.0000028
x: 1.2271846 y: 0.7697921 y*: 0.7698243 d*: 0.0000321
x: 1.3253594 y: 0.7647207 y*: 0.7647813 d*: 0.0000606
x: 1.4235342 y: 0.7490132 y*: 0.7491009 d*: 0.0000877
x: 1.5217089 y: 0.7233747 y*: 0.7234874 d*: 0.0001127
x: 1.6198837 y: 0.6887100 y*: 0.6888452 d*: 0.0001352
x: 1.7180585 y: 0.6461032 y*: 0.6462580 d*: 0.0001547
x: 1.8162333 y: 0.5967939 y*: 0.5969646 d*: 0.0001708
x: 1.9144080 y: 0.5421488 y*: 0.5423318 d*: 0.0001830
x: 2.0125828 y: 0.4836321 y*: 0.4838233 d*: 0.0001912
x: 2.1107576 y: 0.4227729 y*: 0.4229681 d*: 0.0001952
x: 2.2089323 y: 0.3611312 y*: 0.3613261 d*: 0.0001949
x: 2.3071071 y: 0.3002640 y*: 0.3004543 d*: 0.0001903
x: 2.4052819 y: 0.2416907 y*: 0.2418723 d*: 0.0001815
x: 2.5034566 y: 0.1868600 y*: 0.1870287 d*: 0.0001687
x: 2.6016314 y: 0.1371178 y*: 0.1372702 d*: 0.0001524
x: 2.6998062 y: 0.0936779 y*: 0.0938097 d*: 0.0001319
x: 2.7979810 y: 0.0575953 y*: 0.0577074 d*: 0.0001121
x: 2.8961557 y: 0.0297434 y*: 0.0298195 d*: 0.0000762
x: 2.9943305 y: 0.0107942 y*: 0.0108836 d*: 0.0000894
x: 3.0925053 y: 0.0012042 y*: 0.0007870 d*: 0.0004172

  \end{minted}

\end{document}
