% !TeX TXS-program:bibliography = txs:///biber
\documentclass[14pt, russian]{scrartcl}
\let\counterwithout\relax
\let\counterwithin\relax
%\usepackage{lmodern}
\usepackage{float}
\usepackage{xcolor}
\usepackage{extsizes}
\usepackage{subfig}
\usepackage[export]{adjustbox}
\usepackage{tocvsec2} % возможность менять учитываемую глубину разделов в оглавлении
\usepackage[subfigure]{tocloft}
\usepackage[newfloat,outputdir=build]{minted}
\captionsetup[listing]{position=top}

\AtBeginEnvironment{figure}{\vspace{0.5cm}}
\AtBeginEnvironment{table}{\vspace{0.5cm}}
\AtBeginEnvironment{listing}{\vspace{0.5cm}}
\AtBeginEnvironment{algorithm}{\vspace{0.5cm}}
\AtBeginEnvironment{minted}{\vspace{-0.5cm}}

\usepackage{fancyvrb}
\usepackage{ulem,bm,mathrsfs,ifsym} %зачеркивания, особо жирный стиль и RSFS начертание
\usepackage{sectsty} % переопределение стилей подразделов
%%%%%%%%%%%%%%%%%%%%%%%

%%% Поля и разметка страницы %%%
\usepackage{pdflscape}                              % Для включения альбомных страниц
\usepackage{geometry}                               % Для последующего задания полей
\geometry{a4paper,tmargin=2cm,bmargin=2cm,lmargin=3cm,rmargin=1cm} % тоже самое, но лучше

%%% Математические пакеты %%%
\usepackage{amsthm,amsfonts,amsmath,amssymb,amscd}  % Математические дополнения от AMS
\usepackage{mathtools}                              % Добавляет окружение multlined
\usepackage[perpage]{footmisc}
%\usepackage{times}

%%%% Установки для размера шрифта 14 pt %%%%
%% Формирование переменных и констант для сравнения (один раз для всех подключаемых файлов)%%
%% должно располагаться до вызова пакета fontspec или polyglossia, потому что они сбивают его работу
%\newlength{\curtextsize}
%\newlength{\bigtextsize}
%\setlength{\bigtextsize}{13pt}
\KOMAoptions{fontsize=14pt}

\makeatletter
\def\showfontsize{\f@size{} point}
\makeatother

%\makeatletter
%\show\f@size                                       % неплохо для отслеживания, но вызывает стопорение процесса, если документ компилируется без команды  -interaction=nonstopmode
%\setlength{\curtextsize}{\f@size pt}
%\makeatother

%шрифт times
\usepackage{tempora}
%\usepackage{pscyr}
%\setmainfont[Ligatures={TeX,Historic}]{Times New Roman}

   %%% Решение проблемы копирования текста в буфер кракозябрами
%    \input glyphtounicode.tex
%    \input glyphtounicode-cmr.tex %from pdfx package
%    \pdfgentounicode=1
    \usepackage{cmap}                               % Улучшенный поиск русских слов в полученном pdf-файле
    \usepackage[T1]{fontenc}                       % Поддержка русских букв
    \usepackage[utf8]{inputenc}                     % Кодировка utf8
    \usepackage[english, main=russian]{babel}            % Языки: русский, английский
%   \IfFileExists{pscyr.sty}{\usepackage{pscyr}}{}  % Красивые русские шрифты
%\renewcommand{\rmdefault}{ftm}
%%% Оформление абзацев %%%
\usepackage{indentfirst}                            % Красная строка
%\usepackage{eskdpz}

%%% Таблицы %%%
\usepackage{longtable}                              % Длинные таблицы
\usepackage{multirow,makecell,array}                % Улучшенное форматирование таблиц
\usepackage{booktabs}                               % Возможность оформления таблиц в классическом книжном стиле (при правильном использовании не противоречит ГОСТ)

%%% Общее форматирование
\usepackage{soulutf8}                               % Поддержка переносоустойчивых подчёркиваний и зачёркиваний
\usepackage{icomma}                                 % Запятая в десятичных дробях



%%% Изображения %%%
\usepackage{graphicx}                               % Подключаем пакет работы с графикой
\usepackage{wrapfig}

\usepackage{tikz}
\usetikzlibrary{shapes.misc}
\usetikzlibrary{trees}

%%% Списки %%%
\usepackage{enumitem}

%%% Подписи %%%
\usepackage{caption}                                % Для управления подписями (рисунков и таблиц) % Может управлять номерами рисунков и таблиц с caption %Иногда может управлять заголовками в списках рисунков и таблиц
%% Использование:
%\begin{table}[h!]\ContinuedFloat - чтобы не переключать счетчик
%\captionsetup{labelformat=continued}% должен стоять до самого caption
%\caption{}
% либо ручками \caption*{Продолжение таблицы~\ref{...}.} :)

%%% Интервалы %%%
\addto\captionsrussian{%
  \renewcommand{\listingname}{Листинг}%
}
%%% Счётчики %%%
\usepackage[figure,table,section]{totalcount}               % Счётчик рисунков и таблиц
\DeclareTotalCounter{lstlisting}
\usepackage{totcount}                               % Пакет создания счётчиков на основе последнего номера подсчитываемого элемента (может требовать дважды компилировать документ)
\usepackage{totpages}                               % Счётчик страниц, совместимый с hyperref (ссылается на номер последней страницы). Желательно ставить последним пакетом в преамбуле

%%% Продвинутое управление групповыми ссылками (пока только формулами) %%%
%% Кодировки и шрифты %%%

%   \newfontfamily{\cyrillicfont}{Times New Roman}
%   \newfontfamily{\cyrillicfonttt}{CMU Typewriter Text}
	%\setmainfont{Times New Roman}
	%\newfontfamily\cyrillicfont{Times New Roman}
	%\setsansfont{Times New Roman}                    %% задаёт шрифт без засечек
%	\setmonofont{Liberation Mono}               %% задаёт моноширинный шрифт
%    \IfFileExists{pscyr.sty}{\renewcommand{\rmdefault}{ftm}}{}
%%% Интервалы %%%
%linespread-реализация ближе к реализации полуторного интервала в ворде.
%setspace реализация заточена под шрифты 10, 11, 12pt, под остальные кегли хуже, но всё же ближе к типографской классике.
\linespread{1.3}                    % Полуторный интервал (ГОСТ Р 7.0.11-2011, 5.3.6)
%\renewcommand{\@biblabel}[1]{#1}

%%% Гиперссылки %%%
\usepackage{hyperref}

%%% Выравнивание и переносы %%%
\sloppy                             % Избавляемся от переполнений
\clubpenalty=10000                  % Запрещаем разрыв страницы после первой строки абзаца
\widowpenalty=10000                 % Запрещаем разрыв страницы после последней строки абзаца

\makeatletter % малые заглавные, small caps shape
\let\@@scshape=\scshape
\renewcommand{\scshape}{%
  \ifnum\strcmp{\f@series}{bx}=\z@
    \usefont{T1}{cmr}{bx}{sc}%
  \else
    \ifnum\strcmp{\f@shape}{it}=\z@
      \fontshape{scsl}\selectfont
    \else
      \@@scshape
    \fi
  \fi}
\makeatother

%%% Подписи %%%
%\captionsetup{%
%singlelinecheck=off,                % Многострочные подписи, например у таблиц
%skip=2pt,                           % Вертикальная отбивка между подписью и содержимым рисунка или таблицы определяется ключом
%justification=centering,            % Центрирование подписей, заданных командой \caption
%}
%%%        Подключение пакетов                 %%%
\usepackage{ifthen}                 % добавляет ifthenelse
%%% Инициализирование переменных, не трогать!  %%%
\newcounter{intvl}
\newcounter{otstup}
\newcounter{contnumeq}
\newcounter{contnumfig}
\newcounter{contnumtab}
\newcounter{pgnum}
\newcounter{bibliosel}
\newcounter{chapstyle}
\newcounter{headingdelim}
\newcounter{headingalign}
\newcounter{headingsize}
\newcounter{tabcap}
\newcounter{tablaba}
\newcounter{tabtita}
%%%%%%%%%%%%%%%%%%%%%%%%%%%%%%%%%%%%%%%%%%%%%%%%%%

%%% Область упрощённого управления оформлением %%%

%% Интервал между заголовками и между заголовком и текстом
% Заголовки отделяют от текста сверху и снизу тремя интервалами (ГОСТ Р 7.0.11-2011, 5.3.5)
\setcounter{intvl}{3}               % Коэффициент кратности к размеру шрифта

%% Отступы у заголовков в тексте
\setcounter{otstup}{0}              % 0 --- без отступа; 1 --- абзацный отступ

%% Нумерация формул, таблиц и рисунков
\setcounter{contnumeq}{1}           % Нумерация формул: 0 --- пораздельно (во введении подряд, без номера раздела); 1 --- сквозная нумерация по всей диссертации
\setcounter{contnumfig}{1}          % Нумерация рисунков: 0 --- пораздельно (во введении подряд, без номера раздела); 1 --- сквозная нумерация по всей диссертации
\setcounter{contnumtab}{1}          % Нумерация таблиц: 0 --- пораздельно (во введении подряд, без номера раздела); 1 --- сквозная нумерация по всей диссертации

%% Оглавление
\setcounter{pgnum}{0}               % 0 --- номера страниц никак не обозначены; 1 --- Стр. над номерами страниц (дважды компилировать после изменения)

%% Библиография
\setcounter{bibliosel}{1}           % 0 --- встроенная реализация с загрузкой файла через движок bibtex8; 1 --- реализация пакетом biblatex через движок biber

%% Текст и форматирование заголовков
\setcounter{chapstyle}{1}           % 0 --- разделы только под номером; 1 --- разделы с названием "Глава" перед номером
\setcounter{headingdelim}{1}        % 0 --- номер отделен пропуском в 1em или \quad; 1 --- номера разделов и приложений отделены точкой с пробелом, подразделы пропуском без точки; 2 --- номера разделов, подразделов и приложений отделены точкой с пробелом.

%% Выравнивание заголовков в тексте
\setcounter{headingalign}{0}        % 0 --- по центру; 1 --- по левому краю

%% Размеры заголовков в тексте
\setcounter{headingsize}{0}         % 0 --- по ГОСТ, все всегда 14 пт; 1 --- пропорционально изменяющийся размер в зависимости от базового шрифта

%% Подпись таблиц
\setcounter{tabcap}{0}              % 0 --- по ГОСТ, номер таблицы и название разделены тире, выровнены по левому краю, при необходимости на нескольких строках; 1 --- подпись таблицы не по ГОСТ, на двух и более строках, дальнейшие настройки:
%Выравнивание первой строки, с подписью и номером
\setcounter{tablaba}{2}             % 0 --- по левому краю; 1 --- по центру; 2 --- по правому краю
%Выравнивание строк с самим названием таблицы
\setcounter{tabtita}{1}             % 0 --- по левому краю; 1 --- по центру; 2 --- по правому краю

%%% Рисунки %%%
\DeclareCaptionLabelSeparator*{emdash}{~--- }             % (ГОСТ 2.105, 4.3.1)
\captionsetup[figure]{labelsep=emdash,font=onehalfspacing,position=bottom}

%%% Таблицы %%%
\ifthenelse{\equal{\thetabcap}{0}}{%
    \newcommand{\tabcapalign}{\raggedright}  % по левому краю страницы или аналога parbox
}

\ifthenelse{\equal{\thetablaba}{0} \AND \equal{\thetabcap}{1}}{%
    \newcommand{\tabcapalign}{\raggedright}  % по левому краю страницы или аналога parbox
}

\ifthenelse{\equal{\thetablaba}{1} \AND \equal{\thetabcap}{1}}{%
    \newcommand{\tabcapalign}{\centering}    % по центру страницы или аналога parbox
}

\ifthenelse{\equal{\thetablaba}{2} \AND \equal{\thetabcap}{1}}{%
    \newcommand{\tabcapalign}{\raggedleft}   % по правому краю страницы или аналога parbox
}

\ifthenelse{\equal{\thetabtita}{0} \AND \equal{\thetabcap}{1}}{%
    \newcommand{\tabtitalign}{\raggedright}  % по левому краю страницы или аналога parbox
}

\ifthenelse{\equal{\thetabtita}{1} \AND \equal{\thetabcap}{1}}{%
    \newcommand{\tabtitalign}{\centering}    % по центру страницы или аналога parbox
}

\ifthenelse{\equal{\thetabtita}{2} \AND \equal{\thetabcap}{1}}{%
    \newcommand{\tabtitalign}{\raggedleft}   % по правому краю страницы или аналога parbox
}

\DeclareCaptionFormat{tablenocaption}{\tabcapalign #1\strut}        % Наименование таблицы отсутствует
\ifthenelse{\equal{\thetabcap}{0}}{%
    \DeclareCaptionFormat{tablecaption}{\tabcapalign #1#2#3}
    \captionsetup[table]{labelsep=emdash}                       % тире как разделитель идентификатора с номером от наименования
}{%
    \DeclareCaptionFormat{tablecaption}{\tabcapalign #1#2\par%  % Идентификатор таблицы на отдельной строке
        \tabtitalign{#3}}                                       % Наименование таблицы строкой ниже
    \captionsetup[table]{labelsep=space}                        % пробельный разделитель идентификатора с номером от наименования
}
\captionsetup[table]{format=tablecaption,singlelinecheck=off,font=onehalfspacing,position=top,skip=-5pt}  % многострочные наименования и прочее
\DeclareCaptionLabelFormat{continued}{Продолжение таблицы~#2}
\setlength{\belowcaptionskip}{.2cm}
\setlength{\intextsep}{0ex}

%%% Подписи подрисунков %%%
\renewcommand{\thesubfigure}{\asbuk{subfigure}}           % Буквенные номера подрисунков
\captionsetup[subfigure]{font={normalsize},               % Шрифт подписи названий подрисунков (не отличается от основного)
    labelformat=brace,                                    % Формат обозначения подрисунка
    justification=centering,                              % Выключка подписей (форматирование), один из вариантов
}
%\DeclareCaptionFont{font12pt}{\fontsize{12pt}{13pt}\selectfont} % объявляем шрифт 12pt для использования в подписях, тут же надо интерлиньяж объявлять, если не наследуется
%\captionsetup[subfigure]{font={font12pt}}                 % Шрифт подписи названий подрисунков (всегда 12pt)

%%% Настройки гиперссылок %%%

\definecolor{linkcolor}{rgb}{0.0,0,0}
\definecolor{citecolor}{rgb}{0,0.0,0}
\definecolor{urlcolor}{rgb}{0,0,0}

\hypersetup{
    linktocpage=true,           % ссылки с номера страницы в оглавлении, списке таблиц и списке рисунков
%    linktoc=all,                % both the section and page part are links
%    pdfpagelabels=false,        % set PDF page labels (true|false)
    plainpages=true,           % Forces page anchors to be named by the Arabic form  of the page number, rather than the formatted form
    colorlinks,                 % ссылки отображаются раскрашенным текстом, а не раскрашенным прямоугольником, вокруг текста
    linkcolor={linkcolor},      % цвет ссылок типа ref, eqref и подобных
    citecolor={citecolor},      % цвет ссылок-цитат
    urlcolor={urlcolor},        % цвет гиперссылок
    pdflang={ru},
}
\urlstyle{same}
%%% Шаблон %%%
%\DeclareRobustCommand{\todo}{\textcolor{red}}       % решаем проблему превращения названия цвета в результате \MakeUppercase, http://tex.stackexchange.com/a/187930/79756 , \DeclareRobustCommand protects \todo from expanding inside \MakeUppercase
\setlength{\parindent}{2.5em}                       % Абзацный отступ. Должен быть одинаковым по всему тексту и равен пяти знакам (ГОСТ Р 7.0.11-2011, 5.3.7).

%%% Списки %%%
% Используем дефис для ненумерованных списков (ГОСТ 2.105-95, 4.1.7)
%\renewcommand{\labelitemi}{\normalfont\bfseries~{---}}
\renewcommand{\labelitemi}{\bfseries~{---}}
\setlist{nosep,%                                    % Единый стиль для всех списков (пакет enumitem), без дополнительных интервалов.
    labelindent=\parindent,leftmargin=*%            % Каждый пункт, подпункт и перечисление записывают с абзацного отступа (ГОСТ 2.105-95, 4.1.8)
}
%%%%%%%%%%%%%%%%%%%%%%
%\usepackage{xltxtra} % load xunicode

\usepackage{ragged2e}
\usepackage[explicit]{titlesec}
\usepackage{placeins}
\usepackage{xparse}
\usepackage{csquotes}

\usepackage{listingsutf8}
\usepackage{url} %пакеты расширений
\usepackage{algorithm, algorithmicx}
\usepackage[noend]{algpseudocode}
\usepackage{blkarray}
\usepackage{chngcntr}
\usepackage{tabularx}
\usepackage[backend=biber,
    bibstyle=gost-numeric,
    citestyle=nature]{biblatex}
\newcommand*\template[1]{\text{<}#1\text{>}}
\addbibresource{biblio.bib}

\titleformat{name=\section,numberless}[block]{\normalfont\Large\centering}{}{0em}{#1}
\titleformat{\section}[block]{\normalfont\Large\bfseries\raggedright}{}{0em}{\thesection\hspace{0.25em}#1}
\titleformat{\subsection}[block]{\normalfont\Large\bfseries\raggedright}{}{0em}{\thesubsection\hspace{0.25em}#1}
\titleformat{\subsubsection}[block]{\normalfont\large\bfseries\raggedright}{}{0em}{\thesubsubsection\hspace{0.25em}#1}

\let\Algorithm\algorithm
\renewcommand\algorithm[1][]{\Algorithm[#1]\setstretch{1.5}}
%\renewcommand{\listingscaption}{Листинг}

\usepackage{pifont}
\usepackage{calc}
\usepackage{suffix}
\usepackage{csquotes}
\DeclareQuoteStyle{russian}
    {\guillemotleft}{\guillemotright}[0.025em]
    {\quotedblbase}{\textquotedblleft}
\ExecuteQuoteOptions{style=russian}
\newcommand{\enq}[1]{\enquote{#1}}
\newcommand{\eng}[1]{\begin{english}#1\end{english}}
% Подчиненные счетчики в окружениях http://old.kpfu.ru/journals/izv_vuz/arch/sample1251.tex
\newcounter{cTheorem}
\newcounter{cDefinition}
\newcounter{cConsequent}
\newcounter{cExample}
\newcounter{cLemma}
\newcounter{cConjecture}
\newtheorem{Theorem}{Теорема}[cTheorem]
\newtheorem{Definition}{Определение}[cDefinition]
\newtheorem{Consequent}{Следствие}[cConsequent]
\newtheorem{Example}{Пример}[cExample]
\newtheorem{Lemma}{Лемма}[cLemma]
\newtheorem{Conjecture}{Гипотеза}[cConjecture]

\renewcommand{\theTheorem}{\arabic{Theorem}}
\renewcommand{\theDefinition}{\arabic{Definition}}
\renewcommand{\theConsequent}{\arabic{Consequent}}
\renewcommand{\theExample}{\arabic{Example}}
\renewcommand{\theLemma}{\arabic{Lemma}}
\renewcommand{\theConjecture}{\arabic{Conjecture}}
%\makeatletter
\NewDocumentCommand{\Newline}{}{\text{\\}}
\newcommand{\sequence}[2]{\ensuremath \left(#1,\ \dots,\ #2\right)}

\definecolor{mygreen}{rgb}{0,0.6,0}
\definecolor{mygray}{rgb}{0.5,0.5,0.5}
\definecolor{mymauve}{rgb}{0.58,0,0.82}
\renewcommand{\listalgorithmname}{Список алгоритмов}
\floatname{algorithm}{Листинг}
\renewcommand{\lstlistingname}{Листинг}
\renewcommand{\thealgorithm}{\arabic{algorithm}}

\newcommand{\refAlgo}[1]{(листинг~\ref{#1})}
\newcommand{\refImage}[1]{(рисунок~\ref{#1})}

\renewcommand{\theenumi}{\arabic{enumi}.}% Меняем везде перечисления на цифра.цифра
\renewcommand{\labelenumi}{\arabic{enumi}.}% Меняем везде перечисления на цифра.цифра
\renewcommand{\theenumii}{\arabic{enumii}}% Меняем везде перечисления на цифра.цифра
\renewcommand{\labelenumii}{(\arabic{enumii})}% Меняем везде перечисления на цифра.цифра
\renewcommand{\theenumiii}{\roman{enumiii}}% Меняем везде перечисления на цифра.цифра
\renewcommand{\labelenumiii}{(\roman{enumiii})}% Меняем везде перечисления на цифра.цифра
%\newfontfamily\AnkaCoder[Path=src/fonts/]{AnkaCoder-r.ttf}
\renewcommand{\labelitemi}{---}
\renewcommand{\labelitemii}{---}

%\usepackage{courier}

\lstdefinelanguage{Refal}{
  alsodigit = {.,<,>},
  morekeywords = [1]{$ENTRY},
  morekeywords = [2]{Go, Put, Get, Open, Close, Arg, Add, Sub, Mul, Div, Symb, Explode, Implode},
  %keyword4
  morekeywords = [3]{<,>},
  %keyword5
  morekeywords = [4]{e.,t.,s.},
  sensitive = true,
  morecomment = [l]{*},
  morecomment = [s]{/*}{*/},
  commentstyle = \color{mygreen},
  morestring = [b]",
  morestring = [b]',
  stringstyle = \color{purple}
}

\makeatletter
\def\p@subsection{}
\def\p@subsubsection{\thesection\,\thesubsection\,}
\makeatother
\newcommand{\prog}[1]{{\ttfamily\small#1}}
\lstset{ %
  backgroundcolor=\color{white},   % choose the background color; you must add \usepackage{color} or \usepackage{xcolor}
  basicstyle=\ttfamily\footnotesize,
  %basicstyle=\footnotesize\AnkaCoder,        % the size of the fonts that are used for the code
  breakatwhitespace=false,         % sets if automatic breaks shoulbd only happen at whitespace
  breaklines=true,                 % sets automatic line breaking
  captionpos=top,                    % sets the caption-position to bottom
  commentstyle=\color{mygreen},    % comment style
  deletekeywords={...},            % if you want to delete keywords from the given language
  escapeinside={\%*}{*)},          % if you want to add LaTeX within your code
  extendedchars=true,              % lets you use non-ASCII characters; for 8-bits encodings only, does not work with UTF-8
  inputencoding=utf8,
  frame=single,                    % adds a frame around the code
  keepspaces=true,                 % keeps spaces in text, useful for keeping indentation of code (possibly needs columns=flexible)
  keywordstyle=\bf,       % keyword style
  language=Refal,                    % the language of the code
  morekeywords={<,>,$ENTRY,Go,Arg, Open, Close, e., s., t., Get, Put},
  							       % if you want to add more keywords to the set
  numbers=left,                    % where to put the line-numbers; possible values are (none, left, right)
  numbersep=5pt,                   % how far the line-numbers are from the code
  xleftmargin=25pt,
  xrightmargin=25pt,
  numberstyle=\small\color{black}, % the style that is used for the line-numbers
  rulecolor=\color{black},         % if not set, the frame-color may be changed on line-breaks within not-black text (e.g. comments (green here))
  showspaces=false,                % show spaces everywhere adding particular underscores; it overrides 'showstringspaces'
  showstringspaces=false,          % underline spaces within strings only
  showtabs=false,                  % show tabs within strings adding particular underscores
  stepnumber=1,                    % the step between two line-numbers. If it's 1, each line will be numbered
  stringstyle=\color{mymauve},     % string literal style
  tabsize=8,                       % sets default tabsize to 8 spaces
  title=\lstname                   % show the filename of files included with \lstinputlisting; also try caption instead of title
}
\newcommand{\anonsection}[1]{\cleardoublepage
\phantomsection
\addcontentsline{toc}{section}{\protect\numberline{}#1}
\section*{#1}\vspace*{2.5ex} % По госту положены 3 пустые строки после заголовка ненумеруемого раздела
}
\newcommand{\sectionbreak}{\clearpage}
\renewcommand{\sectionfont}{\normalsize} % Сбиваем стиль оглавления в стандартный
\renewcommand{\cftsecleader}{\cftdotfill{\cftdotsep}} % Точки в оглавлении напротив разделов

\renewcommand{\cftsecfont}{\normalfont\large} % Переключение на times в содержании
\renewcommand{\cftsubsecfont}{\normalfont\large} % Переключение на times в содержании

\usepackage{caption}
%\captionsetup[table]{justification=raggedleft}
%\captionsetup[figure]{justification=centering,labelsep=endash}
\usepackage{amsmath}    % \bar    (матрицы и проч. ...)
\usepackage{amsfonts}   % \mathbb (символ для множества действительных чисел и проч. ...)
\usepackage{mathtools}  % \abs, \norm
    \DeclarePairedDelimiter\abs{\lvert}{\rvert} % операция модуля
    \DeclarePairedDelimiter\norm{\lVert}{\rVert} % операция нормы
\DeclareTextCommandDefault{\textvisiblespace}{%
  \mbox{\kern.06em\vrule \@height.3ex}%
  \vbox{\hrule \@width.3em}%
  \hbox{\vrule \@height.3ex}}
\newsavebox{\spacebox}
\begin{lrbox}{\spacebox}
\verb*! !
\end{lrbox}
\newcommand{\aspace}{\usebox{\spacebox}}
\DeclareTotalCounter{listing}

\makeatletter
\renewcommand*{\p@subsubsection}{}
\makeatother

\makeatletter
\AddToHook{begindocument/before}{\@ifpackageloaded{minted}{\removefromtoclist[float]{lol}}{}}
\makeatother

\begin{document}
\sloppy

\def\figurename{Рисунок}

\begin{titlepage}
	\thispagestyle{empty}
	\newpage

	\vspace*{-30pt}
	\hspace{-45pt}
	\begin{minipage}{0.17\textwidth}
		\hspace*{-20pt}\centering
		\includegraphics[width=1.3\textwidth]{emblem.png}
	\end{minipage}
	\begin{minipage}{0.82\textwidth}\small \textbf{
			\vspace*{-0.7ex}
			\hspace*{-10pt}\centerline{Министерство науки и высшего образования Российской Федерации}
			\vspace*{-0.7ex}
			\centerline{Федеральное государственное автономное образовательное учреждение }
			\vspace*{-0.7ex}
			\centerline{высшего образования}
			\vspace*{-0.7ex}
			\centerline{<<Московский государственный технический университет}
			\vspace*{-0.7ex}
			\centerline{имени Н.Э. Баумана}
			\vspace*{-0.7ex}
			\centerline{(национальный исследовательский университет)>>}
			\vspace*{-0.7ex}
			\centerline{(МГТУ им. Н.Э. Баумана)}}
	\end{minipage}

	\vspace{-2pt}
	\hspace{-34.5pt}\rule{\textwidth}{2.5pt}

	\vspace*{-20.3pt}
	\hspace{-34.5pt}\rule{\textwidth}{0.4pt}

	\vspace{0.5ex}
	\noindent \small ФАКУЛЬТЕТ\hspace{80pt} <<Информатика и системы управления>>

	\vspace*{-16pt}
	\hspace{35pt}\rule{0.855\textwidth}{0.4pt}

	\vspace{0.5ex}
	\noindent \small КАФЕДРА\hspace{50pt} <<Теоретическая информатика и компьютерные технологии>>

	\vspace*{-16pt}
	\hspace{25pt}\rule{0.875\textwidth}{0.4pt}


	\vspace{3em}

	\begin{center}
		\textbf{ОТЧЕТ} \\\textit{по лабораторной работе № 4\\по курсу <<Численные методы>>\\на тему: <<Метод Рунге-Кутта численного решения\\ задачи Коши системы ОДУ>>\\Вариант № 26} \\
	\end{center}

	\vspace{\fill}


	\newlength{\ML}
	\settowidth{\ML}{«\underline{\hspace{0.7cm}}» \underline{\hspace{2cm}}}

	\noindent Студент \underline{\text{ИУ9-61Б}} \hfill \underline{ \hspace{4cm}}\quad
	\raisebox{0.45ex}{\underline{\parbox{4cm}{\centering Старовойтов А.И.}}}

	\vspace{-2.1ex}
	\noindent\hspace{9ex}\scriptsize{(Группа)}\normalsize\hspace{170pt}\hspace{2ex}\scriptsize{(Подпись, дата)}\normalsize\hspace{30pt}\hspace{6ex}\scriptsize{(И.О. Фамилия)}\normalsize

	\bigskip

	\noindent Преподаватель  \hfill \underline{\hspace{4cm}}\quad
	\raisebox{0.35ex}{\underline{\parbox{4cm}{\centering Домрачева А.Б.}}}

	\vspace{-2ex}
	\noindent\hspace{13.5ex}\normalsize\hspace{170pt}\hspace{2ex}\scriptsize{(Подпись, дата)}\normalsize\hspace{30pt}\hspace{6ex}\scriptsize{(И.О. Фамилия)}\normalsize

	\bigskip

	%\vspace{\fill}



	\begin{center}
		\textsl{2025 г.}
	\end{center}
\end{titlepage}

%\renewcommand{\ttdefault}{pcr}

\setlength{\tabcolsep}{3pt}
\newpage
\setcounter{page}{2}
%----------------------------------------------------------------------------
%                  ОТСЮДА --- СОБСТВЕННО ТЕКСТ
%----------------------------------------------------------------------------

\newpage
\section{Цель}

Реализация метода Рунге-Кутта численного решения задачи Коши для СОДУ.

\section{Постановка задачи}

Найти численно с погрешностью $\varepsilon$ = 0.001 решение задачи Коши для ДУ второго
порядка, привести его к СОДУ первого порядка, используя метод Рунге-Кутта. Найти
точное решение ДУ. Сравнить приближенное и точное решения на каждом шаге
вычислений.

Индивидуальный вариант: $y'' - 8y' + 7y = 14$, $y(0) = 1$, $y'(0) = 5$ на $[0, 1]$

\section{Теоретический раздел}

Метод Рунге-Кутта применяется для решения задачи Коши для системы ОДУ первого
порядка.

\[
\begin{cases}
y_1' = f_1(x, y_1, y_2, \ldots, y_n) \\
y_2' = f_2(x, y_1, y_2, \ldots, y_n) \\
\vdots \\
y_n' = f_n(x, y_1, y_2, \ldots, y_n)
\end{cases}
\]

на отрезке \([x_0, x_{\text{end}}]\) с начальными условиями:

\[ y_1(x_0) = y_{01}, \ldots, y_n(x_0) = y_{0n} \]

Сам метод заключается в последовательном нахождении вектор-коэффициентов
\(k_1, k_2, k_3\) и \(k_4\) по формулам:

\[
\begin{aligned}
k_1 &= f(x, y) \\
k_2 &= f\left(x + \frac{h}{2}, y + \frac{hk_1}{2}\right) \\
k_3 &= f\left(x + \frac{h}{2}, y + \frac{hk_2}{2}\right) \\
k_4 &= f\left(x + \frac{h}{2}, y + \frac{hk_3}{2}\right)
\end{aligned}
\]

и построении приближения к решению СОДУ в точке \(x + h\):

\[ y(x + h) = y_h = y + \frac{h}{6}(k_1 + 2(k_2 + k_3) + k_4) \]

\[ \text{err} = \frac{1}{2^p - 1}\|y_h - y_{2h}\| \]

\section{Практический раздел}

\captionof{listing}{Исходный код программы на Python}
\vspace{0.5cm}
  \begin{minted}[style=bw, breaklines, frame=single, fontsize = \footnotesize, linenos=false, xleftmargin = 1.5em]{python}
#!/usr/bin/env python3

# Вариант 26

import math


a, b = 0.0, 1.0
eps = 0.001
y0 = 1.0
dy0 = 5.0
n = 100
h = (b-a) / n


def dydx(x, y1, y2):
    dy1 = y2
    dy2 = 8 * y2 - 7 * y1 + 14
    return dy1, dy2

def runge_kutta_step(x, y1, y2, h):
    k1y1, k1y2 = dydx(x, y1, y2)
    k2y1, k2y2 = dydx(x + h/2, y1 + h/2 * k1y1, y2 + h/2 * k1y2)
    k3y1, k3y2 = dydx(x + h/2, y1 + h/2 * k2y1, y2 + h/2 * k2y2)
    k4y1, k4y2 = dydx(x + h, y1 + h * k3y1, y2 + h * k3y2)

    next_y1 = y1 + h * (k1y1 + 2*k2y1 + 2*k3y1 + k4y1) / 6
    next_y2 = y2 + h * (k1y2 + 2*k2y2 + 2*k3y2 + k4y2) / 6
    return next_y1, next_y2

def exact_solution(x):
    C1 = 1
    C2 = -2
    return C1 * math.exp(7*x) + C2 * math.exp(x) + 2

x_values = list()
y_num = list()
dy_num = list()
y_exact = list()

x = a
y1 = y0
y2 = dy0

while x <= b:
    x_values.append(x)
    y_num.append(y1)
    y_exact.append(exact_solution(x))
    dy_num.append(y2)

    y1, y2 = runge_kutta_step(x, y1, y2, h)
    x += h

print('|       x         |       y\'      |       y\'\'     |       y\'точное |       delta    |')
for i in range(len(x_values)):
    print('|{:14.7f}   |{:14.7f} |{:14.7f} |{:14.7f}  |{:14.7f}  |'.format(x_values[i], y_num[i], dy_num[i], y_exact[i], abs(y_exact[i] - y_num[i])))
  \end{minted}

\section{Тестирование}

\captionof{listing}{Результаты работы программы}
\vspace{0.5cm}
  \begin{minted}[style=bw, breaklines, frame=single, fontsize = \tiny, linenos=false, xleftmargin = 1.5em]{md}
|       x         |       y'      |       y''     |       y'точное |       delta    |
|     0.0000000   |     1.0000000 |     5.0000000 |     1.0000000  |     0.0000000  |
|     0.0100000   |     1.0524078 |     5.4874568 |     1.0524078  |     0.0000000  |
|     0.0200000   |     1.1098711 |     6.0115137 |     1.1098711  |     0.0000000  |
|     0.0300000   |     1.1727689 |     6.5748370 |     1.1727690  |     0.0000000  |
|     0.0400000   |     1.2415082 |     7.1802866 |     1.2415083  |     0.0000001  |
|     0.0500000   |     1.3165253 |     7.8309300 |     1.3165254  |     0.0000001  |
|     0.0600000   |     1.3982883 |     8.5300570 |     1.3982885  |     0.0000001  |
|     0.0700000   |     1.4872997 |     9.2811961 |     1.4872999  |     0.0000002  |
|     0.0800000   |     1.5840982 |    10.0881321 |     1.5840984  |     0.0000002  |
|     0.0900000   |     1.6892618 |    10.9549239 |     1.6892620  |     0.0000002  |
|     0.1000000   |     1.8034106 |    11.8859253 |     1.8034109  |     0.0000003  |
|     0.1100000   |     1.9272098 |    12.8858054 |     1.9272101  |     0.0000003  |
|     0.1200000   |     2.0613729 |    13.9595726 |     2.0613733  |     0.0000004  |
|     0.1300000   |     2.2066653 |    15.1125980 |     2.2066658  |     0.0000004  |
|     0.1400000   |     2.3639082 |    16.3506426 |     2.3639086  |     0.0000005  |
|     0.1500000   |     2.5339821 |    17.6798854 |     2.5339826  |     0.0000006  |
|     0.1600000   |     2.7178318 |    19.1069531 |     2.7178325  |     0.0000006  |
|     0.1700000   |     2.9164708 |    20.6389536 |     2.9164715  |     0.0000007  |
|     0.1800000   |     3.1309859 |    22.2835098 |     3.1309868  |     0.0000008  |
|     0.1900000   |     3.3625432 |    24.0487979 |     3.3625442  |     0.0000009  |
|     0.2000000   |     3.6123934 |    25.9435868 |     3.6123945  |     0.0000011  |
|     0.2100000   |     3.8818778 |    27.9772814 |     3.8818790  |     0.0000012  |
|     0.2200000   |     4.1724355 |    30.1599689 |     4.1724368  |     0.0000014  |
|     0.2300000   |     4.4856097 |    32.5024679 |     4.4856112  |     0.0000015  |
|     0.2400000   |     4.8230560 |    35.0163816 |     4.8230577  |     0.0000017  |
|     0.2500000   |     5.1865499 |    37.7141546 |     5.1865518  |     0.0000019  |
|     0.2600000   |     5.5779962 |    40.6091341 |     5.5779983  |     0.0000021  |
|     0.2700000   |     5.9994374 |    43.7156353 |     5.9994398  |     0.0000024  |
|     0.2800000   |     6.4530648 |    47.0490114 |     6.4530674  |     0.0000026  |
|     0.2900000   |     6.9412285 |    50.6257291 |     6.9412314  |     0.0000029  |
|     0.3000000   |     7.4664491 |    54.4634491 |     7.4664523  |     0.0000032  |
|     0.3100000   |     8.0314302 |    58.5811129 |     8.0314338  |     0.0000036  |
|     0.3200000   |     8.6390718 |    62.9990357 |     8.6390758  |     0.0000040  |
|     0.3300000   |     9.2924840 |    67.7390056 |     9.2924884  |     0.0000044  |
|     0.3400000   |     9.9950028 |    72.8243909 |     9.9950077  |     0.0000049  |
|     0.3500000   |    10.7502063 |    78.2802544 |    10.7502116  |     0.0000054  |
|     0.3600000   |    11.5619319 |    84.1334764 |    11.5619378  |     0.0000059  |
|     0.3700000   |    12.4342959 |    90.4128864 |    12.4343024  |     0.0000065  |
|     0.3800000   |    13.3717127 |    97.1494043 |    13.3717199  |     0.0000072  |
|     0.3900000   |    14.3789175 |   104.3761922 |    14.3789254  |     0.0000079  |
|     0.4000000   |    15.4609887 |   112.1288172 |    15.4609974  |     0.0000087  |
|     0.4100000   |    16.6233731 |   120.4454249 |    16.6233826  |     0.0000096  |
|     0.4200000   |    17.8719127 |   129.3669276 |    17.8719232  |     0.0000105  |
|     0.4300000   |    19.2128734 |   138.9372037 |    19.2128849  |     0.0000115  |
|     0.4400000   |    20.6529753 |   149.2033138 |    20.6529880  |     0.0000126  |
|     0.4500000   |    22.1994263 |   160.2157306 |    22.1994402  |     0.0000139  |
|     0.4600000   |    23.8599570 |   172.0285868 |    23.8599722  |     0.0000152  |
|     0.4700000   |    25.6428586 |   184.6999405 |    25.6428753  |     0.0000167  |
|     0.4800000   |    27.5570238 |   198.2920595 |    27.5570421  |     0.0000183  |
|     0.4900000   |    29.6119903 |   212.8717269 |    29.6120103  |     0.0000200  |
|     0.5000000   |    31.8179875 |   228.5105680 |    31.8180094  |     0.0000219  |
|     0.5100000   |    34.1859868 |   245.2854021 |    34.1860108  |     0.0000239  |
|     0.5200000   |    36.7277553 |   263.2786186 |    36.7277814  |     0.0000262  |
|     0.5300000   |    39.4559133 |   282.5785808 |    39.4559419  |     0.0000286  |
|     0.5400000   |    42.3839967 |   303.2800596 |    42.3840280  |     0.0000313  |
|     0.5500000   |    45.5265230 |   325.4846975 |    45.5265572  |     0.0000341  |
|     0.5600000   |    48.8990625 |   349.3015074 |    48.8990998  |     0.0000373  |
|     0.5700000   |    52.5183146 |   374.8474065 |    52.5183553  |     0.0000407  |
|     0.5800000   |    56.4021898 |   402.2477897 |    56.4022342  |     0.0000444  |
|     0.5900000   |    60.5698976 |   431.6371444 |    60.5699461  |     0.0000485  |
|     0.6000000   |    65.0420406 |   463.1597096 |    65.0420934  |     0.0000529  |
|     0.6100000   |    69.8407152 |   496.9701830 |    69.8407728  |     0.0000576  |
|     0.6200000   |    74.9896204 |   533.2344794 |    74.9896833  |     0.0000628  |
|     0.6300000   |    80.5141739 |   572.1305440 |    80.5142423  |     0.0000685  |
|     0.6400000   |    86.4416363 |   613.8492247 |    86.4417109  |     0.0000746  |
|     0.6500000   |    92.8012454 |   658.5952076 |    92.8013267  |     0.0000813  |
|     0.6600000   |    99.6243590 |   706.5880207 |    99.6244475  |     0.0000885  |
|     0.6700000   |   106.9446088 |   758.0631095 |   106.9447052  |     0.0000964  |
|     0.6800000   |   114.7980655 |   813.2729916 |   114.7981704  |     0.0001049  |
|     0.6900000   |   123.2234154 |   872.4884944 |   123.2235296  |     0.0001142  |
|     0.7000000   |   132.2621501 |   936.0000829 |   132.2622743  |     0.0001242  |
|     0.7100000   |   141.9587697 |  1004.1192833 |   141.9589049  |     0.0001351  |
|     0.7200000   |   152.3610017 |  1077.1802101 |   152.3611486  |     0.0001470  |
|     0.7300000   |   163.5200339 |  1155.5412043 |   163.5201937  |     0.0001598  |
|     0.7400000   |   175.4907662 |  1239.5865898 |   175.4909400  |     0.0001737  |
|     0.7500000   |   188.3320796 |  1329.7285573 |   188.3322684  |     0.0001888  |
|     0.7600000   |   202.1071243 |  1426.4091848 |   202.1073296  |     0.0002052  |
|     0.7700000   |   216.8836300 |  1530.1026053 |   216.8838530  |     0.0002230  |
|     0.7800000   |   232.7342375 |  1641.3173300 |   232.7344798  |     0.0002423  |
|     0.7900000   |   249.7368550 |  1760.5987420 |   249.7371182  |     0.0002632  |
|     0.8000000   |   267.9750397 |  1888.5317692 |   267.9753256  |     0.0002858  |
|     0.8100000   |   287.5384080 |  2025.7437519 |   287.5387184  |     0.0003104  |
|     0.8200000   |   308.5230743 |  2172.9075180 |   308.5234113  |     0.0003370  |
|     0.8300000   |   331.0321223 |  2330.7446812 |   331.0324882  |     0.0003659  |
|     0.8400000   |   355.1761106 |  2500.0291781 |   355.1765078  |     0.0003971  |
|     0.8500000   |   381.0736144 |  2681.5910628 |   381.0740454  |     0.0004310  |
|     0.8600000   |   408.8518067 |  2876.3205750 |   408.8522743  |     0.0004677  |
|     0.8700000   |   438.6470820 |  3085.1725044 |   438.6475894  |     0.0005074  |
|     0.8800000   |   470.6057250 |  3309.1708711 |   470.6062754  |     0.0005505  |
|     0.8900000   |   504.8846271 |  3549.4139455 |   504.8852242  |     0.0005971  |
|     0.9000000   |   541.6520563 |  3807.0796316 |   541.6527039  |     0.0006476  |
|     0.9100000   |   581.0884816 |  4083.4312414 |   581.0891838  |     0.0007023  |
|     0.9200000   |   623.3874576 |  4379.8236878 |   623.3882190  |     0.0007614  |
|     0.9300000   |   668.7565737 |  4697.7101260 |   668.7573992  |     0.0008255  |
|     0.9400000   |   717.4184715 |  5038.6490775 |   717.4193664  |     0.0008949  |
|     0.9500000   |   769.6119362 |  5404.3120693 |   769.6129062  |     0.0009700  |
|     0.9600000   |   825.5930672 |  5796.4918283 |   825.5941185  |     0.0010513  |
|     0.9700000   |   885.6365336 |  6217.1110690 |   885.6376729  |     0.0011393  |
|     0.9800000   |   950.0369205 |  6668.2319186 |   950.0381550  |     0.0012345  |
|     0.9900000   |  1019.1101732 |  7152.0660260 |  1019.1115107  |     0.0013375  |
  \end{minted}

\section{Вывод}

В ходе выполнения лабораторной работы был реализован метод Рунге-Кутта для
численного решения СОДУ. Реализация была выполнена в виде программы на языке
Python и проведено сравнение приближенного и точного решения. Имея четвертый
порядок точности, метод Рунге-Кутта прост в реализации.

\end{document}
